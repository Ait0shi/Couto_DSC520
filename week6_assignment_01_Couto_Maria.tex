\documentclass{article}\usepackage[]{graphicx}\usepackage[]{xcolor}
% maxwidth is the original width if it is less than linewidth
% otherwise use linewidth (to make sure the graphics do not exceed the margin)
\makeatletter
\def\maxwidth{ %
  \ifdim\Gin@nat@width>\linewidth
    \linewidth
  \else
    \Gin@nat@width
  \fi
}
\makeatother

\definecolor{fgcolor}{rgb}{0.345, 0.345, 0.345}
\newcommand{\hlnum}[1]{\textcolor[rgb]{0.686,0.059,0.569}{#1}}%
\newcommand{\hlstr}[1]{\textcolor[rgb]{0.192,0.494,0.8}{#1}}%
\newcommand{\hlcom}[1]{\textcolor[rgb]{0.678,0.584,0.686}{\textit{#1}}}%
\newcommand{\hlopt}[1]{\textcolor[rgb]{0,0,0}{#1}}%
\newcommand{\hlstd}[1]{\textcolor[rgb]{0.345,0.345,0.345}{#1}}%
\newcommand{\hlkwa}[1]{\textcolor[rgb]{0.161,0.373,0.58}{\textbf{#1}}}%
\newcommand{\hlkwb}[1]{\textcolor[rgb]{0.69,0.353,0.396}{#1}}%
\newcommand{\hlkwc}[1]{\textcolor[rgb]{0.333,0.667,0.333}{#1}}%
\newcommand{\hlkwd}[1]{\textcolor[rgb]{0.737,0.353,0.396}{\textbf{#1}}}%
\let\hlipl\hlkwb

\usepackage{framed}
\makeatletter
\newenvironment{kframe}{%
 \def\at@end@of@kframe{}%
 \ifinner\ifhmode%
  \def\at@end@of@kframe{\end{minipage}}%
  \begin{minipage}{\columnwidth}%
 \fi\fi%
 \def\FrameCommand##1{\hskip\@totalleftmargin \hskip-\fboxsep
 \colorbox{shadecolor}{##1}\hskip-\fboxsep
     % There is no \\@totalrightmargin, so:
     \hskip-\linewidth \hskip-\@totalleftmargin \hskip\columnwidth}%
 \MakeFramed {\advance\hsize-\width
   \@totalleftmargin\z@ \linewidth\hsize
   \@setminipage}}%
 {\par\unskip\endMakeFramed%
 \at@end@of@kframe}
\makeatother

\definecolor{shadecolor}{rgb}{.97, .97, .97}
\definecolor{messagecolor}{rgb}{0, 0, 0}
\definecolor{warningcolor}{rgb}{1, 0, 1}
\definecolor{errorcolor}{rgb}{1, 0, 0}
\newenvironment{knitrout}{}{} % an empty environment to be redefined in TeX

\usepackage{alltt}
\usepackage[sc]{mathpazo}
\renewcommand{\sfdefault}{lmss}
\renewcommand{\ttdefault}{lmtt}
\usepackage[T1]{fontenc}
\usepackage{geometry}
\geometry{verbose,tmargin=2.5cm,bmargin=2.5cm,lmargin=2.5cm,rmargin=2.5cm}
\setcounter{secnumdepth}{2}
\setcounter{tocdepth}{2}
\usepackage[unicode=true,pdfusetitle,
 bookmarks=true,bookmarksnumbered=true,bookmarksopen=true,bookmarksopenlevel=2,
 breaklinks=false,pdfborder={0 0 1},backref=false,colorlinks=false]
 {hyperref}
\hypersetup{
 pdfstartview={XYZ null null 1}}

\makeatletter
%%%%%%%%%%%%%%%%%%%%%%%%%%%%%% User specified LaTeX commands.
\renewcommand{\textfraction}{0.05}
\renewcommand{\topfraction}{0.8}
\renewcommand{\bottomfraction}{0.8}
\renewcommand{\floatpagefraction}{0.75}

\makeatother
\IfFileExists{upquote.sty}{\usepackage{upquote}}{}
\begin{document}








The results below are generated from an R script.

\begin{knitrout}
\definecolor{shadecolor}{rgb}{0.969, 0.969, 0.969}\color{fgcolor}\begin{kframe}
\begin{alltt}
\hlcom{# Load the ggplot2 package}
\hlcom{# Set the working directory to the root of your DSC 520 directory}
\hlcom{# Load the `data/r4ds/heights.csv` to}

\hlkwd{library}(ggplot2)
\hlkwd{theme_set}(\hlkwd{theme_minimal}())
\hlkwd{setwd}(\hlstr{"C:/Users/ait0s/OneDrive/Documents/GitHub/dsc520"})
heights_df <- \hlkwd{read.csv}(\hlstr{"data/r4ds/heights.csv"})
\hlkwd{summary}(heights_df)



\hlcom{# https://ggplot2.tidyverse.org/reference/geom_boxplot.html}
\hlcom{# Create boxplots of sex vs. earn and race vs. earn using `geom_point()` }
\hlcom{# and `geom_boxplot()`}

\hlcom{# sex vs. earn}



\hlkwd{ggplot}(heights_df, \hlkwd{aes}(x=sex, y=earn)) + \hlkwd{geom_point}()+ \hlkwd{geom_boxplot}()


\hlcom{# race vs. earn}

\hlkwd{ggplot}(heights_df, \hlkwd{aes}(x=race, y=earn)) + \hlkwd{geom_point}()+ \hlkwd{geom_boxplot}()


\hlcom{# https://ggplot2.tidyverse.org/reference/geom_bar.html}
\hlcom{# Using `geom_bar()` plot a bar chart of the number of records for each `sex`}



\hlkwd{ggplot}(heights_df, \hlkwd{aes}(sex)) + \hlkwd{geom_bar}()


\hlcom{# Using `geom_bar()` plot a bar chart of the number of records for each race}


\hlkwd{ggplot}(heights_df, \hlkwd{aes}(race)) + \hlkwd{geom_bar}()


\hlcom{# Create a horizontal bar chart by adding `coord_flip()` to the previous plot}


\hlkwd{ggplot}(heights_df, \hlkwd{aes}(race)) + \hlkwd{geom_bar}() + \hlkwd{coord_flip}()


\hlcom{# https://www.rdocumentation.org/packages/ggplot2/versions/3.3.0/topics/geom_path}
\hlcom{# Load the file `"data/nytimes/covid-19-data/us-states.csv"` and}
\hlcom{# assign it to the `covid_df` dataframe}


covid_df <- \hlkwd{read.csv}(\hlstr{"data/nytimes/covid-19-data/us-states.csv"})

\hlkwd{head}(covid_df)


\hlcom{# Parse the date column using `as.Date()`}


covid_df$date <- \hlkwd{as.Date}(covid_df$date)


\hlcom{# Create three dataframes named `california_df`, `ny_df`, and `florida_df`}
\hlcom{# containing the data from California, New York, and Florida}

California

california_df <- covid_df[ \hlkwd{which}( covid_df$state == \hlstr{"California"}), ]

\hlkwd{summary}(california_df)


\hlcom{# New York}

ny_df <- covid_df[ \hlkwd{which}( covid_df$state == \hlstr{"New York"}), ]

\hlkwd{summary}(ny_df)


\hlcom{# Florida}


florida_df <- covid_df[ \hlkwd{which}( covid_df$state == \hlstr{"Florida"}), ]

\hlkwd{summary}(florida_df)


\hlcom{# Plot the number of cases in Florida using `geom_line()`}


\hlkwd{ggplot}(data=florida_df, \hlkwd{aes}(x=date, y=cases, group=1)) + \hlkwd{geom_line}()


Add lines for New York and California to the plot


\hlkwd{ggplot}(data=florida_df, \hlkwd{aes}(x=date, group=1)) +
  \hlkwd{geom_line}(\hlkwd{aes}(y = cases)) +
  \hlkwd{geom_line}(data=ny_df, \hlkwd{aes}(y = cases)) +
  \hlkwd{geom_line}(data=california_df, \hlkwd{aes}(y = cases))


\hlcom{# Use the colors "darkred", "darkgreen", and "steelblue" for Florida, }
\hlcom{# New York, and California}


\hlkwd{ggplot}(data=florida_df, \hlkwd{aes}(x=date, group=1)) +
  \hlkwd{geom_line}(\hlkwd{aes}(y =cases), color = \hlstr{"darkred"}, ) +
  \hlkwd{geom_line}(data=ny_df, \hlkwd{aes}(y = cases), color = \hlstr{"darkgreen"}) +
  \hlkwd{geom_line}(data=california_df, \hlkwd{aes}(y = cases), color = \hlstr{"steelblue"})


\hlcom{# Add a legend to the plot using `scale_colour_manual`}
\hlcom{# Add a blank (" ") label to the x-axis and the label "Cases" to the y axis}


\hlkwd{ggplot}(data=florida_df, \hlkwd{aes}(x=date, group=1)) +
  \hlkwd{geom_line}(\hlkwd{aes}(y = cases, colour = \hlstr{"Florida"})) +
  \hlkwd{geom_line}(data=ny_df, \hlkwd{aes}(y = cases,colour=\hlstr{"New York"})) +
  \hlkwd{geom_line}(data=california_df, \hlkwd{aes}(y = cases, colour=\hlstr{"California"})) +
  \hlkwd{scale_colour_manual}(\hlstr{""},
                      breaks = \hlkwd{c}(\hlstr{"Florida"}, \hlstr{"New York"}, \hlstr{"California"}),
                      values = \hlkwd{c}(\hlstr{"darkred"}, \hlstr{"darkgreen"}, \hlstr{"steelblue"})) +
  \hlkwd{xlab}(\hlstr{" "}) + \hlkwd{ylab}(\hlstr{"Cases"})


\hlcom{# Scale the y axis using `scale_y_log10()`}

\hlkwd{ggplot}(data=florida_df, \hlkwd{aes}(x=date, group=1)) +
  \hlkwd{geom_line}(\hlkwd{aes}(y = cases, colour = \hlstr{"Florida"})) +
  \hlkwd{geom_line}(data=ny_df, \hlkwd{aes}(y = cases,colour=\hlstr{"New York"})) +
  \hlkwd{geom_line}(data=california_df, \hlkwd{aes}(y = cases, colour=\hlstr{"California"})) +
  \hlkwd{scale_colour_manual}(\hlstr{""},
                      breaks = \hlkwd{c}(\hlstr{"Florida"}, \hlstr{"New York"}, \hlstr{"California"}),
                      values = \hlkwd{c}(\hlstr{"darkred"}, \hlstr{"darkgreen"}, \hlstr{"steelblue"})) +
  \hlkwd{xlab}(\hlstr{" "}) + \hlkwd{ylab}(\hlstr{"Cases"}) + \hlkwd{scale_y_log10}()
\end{alltt}


{\ttfamily\noindent\bfseries\color{errorcolor}{\#\# Error: <text>:96:5: unexpected symbol\\\#\# 95: \\\#\# 96: Add lines\\\#\# \ \ \ \ \ \ \ \ \textasciicircum{}}}\end{kframe}
\end{knitrout}

The R session information (including the OS info, R version and all
packages used):

\begin{knitrout}
\definecolor{shadecolor}{rgb}{0.969, 0.969, 0.969}\color{fgcolor}\begin{kframe}
\begin{alltt}
\hlkwd{sessionInfo}\hlstd{()}
\end{alltt}
\begin{verbatim}
## R version 4.3.0 (2023-04-21 ucrt)
## Platform: x86_64-w64-mingw32/x64 (64-bit)
## Running under: Windows 11 x64 (build 22621)
## 
## Matrix products: default
## 
## 
## locale:
## [1] LC_COLLATE=English_United States.utf8  LC_CTYPE=C                            
## [3] LC_MONETARY=English_United States.utf8 LC_NUMERIC=C                          
## [5] LC_TIME=English_United States.utf8    
## 
## time zone: America/New_York
## tzcode source: internal
## 
## attached base packages:
## [1] stats     graphics  grDevices utils     datasets  methods   base     
## 
## other attached packages:
## [1] ggplot2_3.4.2
## 
## loaded via a namespace (and not attached):
##  [1] vctrs_0.6.2      cli_3.6.1        knitr_1.43       rlang_1.1.1      xfun_0.39       
##  [6] highr_0.10       DBI_1.1.3        generics_0.1.3   labeling_0.4.2   glue_1.6.2      
## [11] bit_4.0.5        colorspace_2.1-0 htmltools_0.5.5  fansi_1.0.4      rsconnect_0.8.29
## [16] scales_1.2.1     rmarkdown_2.23   grid_4.3.0       pander_0.6.5     tibble_3.2.1    
## [21] munsell_0.5.0    evaluate_0.21    fastmap_1.1.1    yaml_2.3.7       lifecycle_1.0.3 
## [26] memoise_2.0.1    compiler_4.3.0   dplyr_1.1.2      RSQLite_2.3.1    blob_1.2.4      
## [31] pkgconfig_2.0.3  Rcpp_1.0.10      rstudioapi_0.14  farver_2.1.1     digest_0.6.31   
## [36] R6_2.5.1         tidyselect_1.2.0 utf8_1.2.3       pillar_1.9.0     magrittr_2.0.3  
## [41] withr_2.5.0      tools_4.3.0      bit64_4.0.5      gtable_0.3.3     cachem_1.0.8
\end{verbatim}
\begin{alltt}
\hlkwd{Sys.time}\hlstd{()}
\end{alltt}
\begin{verbatim}
## [1] "2023-07-20 01:08:06 EDT"
\end{verbatim}
\end{kframe}
\end{knitrout}


\end{document}
